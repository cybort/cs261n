\documentclass{article}  
\usepackage{url}

\begin{document}

% VPN arms race We seek to build a system that remains difficult to block even
% after it has many users.

\section{Related work}

In response to Internet censorship, many research papers and pragmatic systems
such Freegate, Psiphon, and Lantern have explored the design space of censorship
circumvention. All these schemes are based on a simple idea: deploy a set of
proxies outside the censor's domain, and let the user connect to one of them
which can pass data on to the true destination for the user.\\

Broadly speaking, there are two main challenges to proxy-based circumvention.
The first is obfuscation: making circumvention traffic difficult to distinguish
from traffic the censor wishes to allow. The second is proxy address
unblockability: making it difficult for the censor simply to block proxies by
address. Past works have considered one or both of these problems.\\

To protect the relays from being detected and blocked, a school of research
tries to use covert channels for the communication with relays.
FreeWave~\cite{freewave} encodes data into audio streams and use the Skype P2P
network to communicate with the proxy located outside the censored domain, which
is essentially another Skype client that decodes the stream. The proxy then
forward the traffic to the true destination. CensorSpoofer~\cite{censorspoofer}
decouples the upstream and downstream channel. It assumes the existence of a
low-bandwidth indirect channel that is immune from censorship. It uses this
channel for send requests (URLs) to the proxy while sending dummy requests to a
dummy host. The proxy then fetches the web content and spoofs the dummy host's
IP address to deliver the content back. A technique known as OSS~\cite{oss}, for
``online scanning service,'' closely resembles ours in important ways. It uses a
third-party web service as a conduit to carry traffic, encoded either in a URL
or in an HTTP body. Because web services cannot in general be assumed to
reproduce a web server's response exactly (a translation service, for example,
returns a response after translating it to another language), OSS does not use
response bodies to carry data, instead requiring the relay to initiate a
reflected request back to the client. OSS also does not claim that any one
service is difficult to block, rather that there are many web services that can
be used. The main ways in which this work differs is that we effectively control
the OSS: we can count on response bodies being preserved and so use them to send
downstream data; and we put the OSS on infrastructure that is expensive to
block.\\

Traffic using covert channels may exhibit distinguishable characteristics. As
noted by Winter et al.~\cite{foci12-winter}, Tor traffic is currently
distinguishable from what is regarded as harmless traffic. In addition, their
experiments show that the Great Firewall of China actively scans for Tor bridges
and block them. Therefore, another school of research focuses on obfuscation
techniques for hiding protocol patterns. One technique is to look unlike
anything forbidden by the censor; that is, fail to match a blacklist. Another is
to resemble a protocol that is explicitly allowed; that is, match a whitelist.
Falling into the first category are ``look-like-nothing'' transports whose
payloads are indistinguishable from a uniformly random byte stream. A examples
of such a transport is obfsproxy~\cite{obfsproxy}, long the go-to obfuscated
transport used by Tor. ScrambleSuit~\cite{scramblesuit} is like obfsproxy in the
content of its payloads, but takes additional steps to obscure its signature of
packet lengths and timing, and is designed to resist active scanning for proxies
(the proxy server says nothing until the client proves knowledge of a shared
secret). Another path to obfsucation is the steganographic approach: look like
something the censor doesn't block. StegoTorus~\cite{stegotorus} and Code Talker
Tunnel (formerly called SkypeMorph)~\cite{skypemorph} camouflage Tor traffic and
attempt to mimic HTTP or Skype traffic. CensorSpoofer also mimics SIP-based VoIP
to hide the upstream channel.\\

Houmansadr et al.~\cite{parrot} evaluate these steganography-based (parrot)
systems and conclude that unobservability by imitation is a ``fundamentally
flawed approach'', since these kind of systems must successfully mimic another
protocol. To fully mimic a complex and sometimes proprietary protocol like Skype
is difficult in that the system must mimic the protocol,  reaction to errors and
network conditions, typical traffic, and implementation-specific artifacts
correctly, otherwise a weak censor can recognize such traffic at a low cost.
Therefore we avoid the parrot approach and use HTTPS as the channel. Geddes et
al.~\cite{acks} demonstrate that even non-parrot systems may be vulnerable to
attacks that disrupt that covert communication while having no effect or less
effect on legitimate traffic. Their examination is specific to VoIP protocols,
where packet losses and duplication are acceptable. The censor then can
deliberately drop or inject ACKs to disrupt the covert channel without causing
much collateral damage.\\

Tor has long faced the problem of its entry relays being blocked. The list of
relays is public, so it easy to block all of them by IP address. Tor
bridges~\cite{tor-blocking} are relays that are not universally known, intended
to serve as entry points for censored users. A system called BridgeDB seeks to
provide a few bridges to anyone who asks, while at the same time making it
difficult to learn the entire list of secret bridges. The plain Tor protocol
remains relatively easy to detect, so BridgeDB is also capable of distributing
obfsproxy and ScrambleSuit bridges.\\

There are a few approaches proposed to address the problem of proxy diversity.
Flash proxy~\cite{flashproxy-pets12} attempts to address the problem  by
conscripting web users as temporary proxies. Proxies last only as long as a web
user stays on a page, so the pool of proxies is constantly changing and
difficult to block. Their approach to proxy unblockability is in a sense
opposite to this work's: where flash proxy uses many, cheap, unreliable proxies,
we use just one high-value proxy, to block which would cause expensive
collateral damage. There exists a prototype transport that combines flash proxy
with obfsproxy~\cite{obfs-flash}, however it is limited because it is not
possible to obfuscate flash proxy's outermost WebSocket layer. Decoy routing is
a recently proposed approach for proxy diversity. Telex~\cite{telex} is an
instance of circumvention systems that leverages decoy routing. Telex relies on
friendly ISPs that deploy Telex stations on paths between censors' networks and
popular, uncensored Internet destinations. Telex stations would divert the
seemingly  innocuous flows to a forbidden site if a special ``tag'' is
presented. Telex will prevail only if a large number of ISPs are cooperative and
willing to deploy such stations. A recent study~\cite{nodirectionhome} on ASes
topology suggests that defeating decoy routing is likely to be expensive for the
censors, if the decoy routers are strategically deployed. Despite that decoy
routing is a sound technical approach, it is still questionable if ISPs are
willing to act against state-level censors. However, the takeaway is that the
censors are unwilling to completely block day-to-day Internet access, which we
can take advantage of.\\

GoAgent~\cite{goagent} is a direct inspiration for our work in its use of App
Engine and Host header--based domain fronting. GoAgent requires users to upload
a personal copy of the server code to App Engine, and App Engine itself makes
HTTP requests, not some proxy farther out. According to a May 2013 survey~\cite
{collateral-freedom},  GoAgent was then the circumvention tool most used in
China, with 35\% of survey respondents having used it in the previous month.
This figure is higher than that of paid~(29\%) and free VPNs~(18\%), and far
above that of other special-purpose tools like Tor~(2.9\%) and Psiphon~(2.5\%).
\\

Winter and Lindskog~\cite{foci12-winter} investigated how China's Great Firewall blocks Tor.
They confirmed an earlier discovery of Wilde~\cite{wilde} that the firewall identifies Tor relays using more than passive monitoring:
it actively probes destination addresses to see if they speak the Tor protocol, and if so,
blocks them for the future.
(At the time, the firewall identified connections for future probing by looking for a distinctive
list of ciphersuites sent by a Tor client in its TLS client hello.) %\cite{bug4744}
The discovery of active probing was the motivation for probing resistance in ScrambleSuit.
\\

% CORDON? Say how meek would be classified in it?

\bibliographystyle{plain}  
\bibliography{related-work}

\end{document}
