\documentclass{article}  
\usepackage{url}

\setlength{\parskip}{\baselineskip}

\begin{document}

% VPN arms race We seek to build a system that remains difficult to block even
% after it has many users.

\section{Related work}

In response to Internet censorship, many research papers and pragmatic systems
such Freegate, Psiphon, and Lantern have explored the design space of censorship
circumvention. All these schemes are based on a simple idea: deploy a set of
proxies outside the censor's domain, and let the user connect to one of them
which can pass data on to the true destination for the user.

Broadly speaking, there are two main challenges to proxy-based circumvention.
The first is obfuscation: making circumvention traffic difficult to distinguish
from traffic the censor wishes to allow. The second is proxy address
unblockability: making it difficult for the censor simply to block proxies by
address. Past works have considered one or both of these problems.

The problem of traffic obfuscation has been approached in many different ways.
They may be classified into two general techniques.
The first is to look unlike
anything forbidden by the censor; that is, fail to match a blacklist. The second is
to resemble a protocol that is explicitly allowed; that is, match a whitelist.
Falling into the first category are ``look-like-nothing'' transports whose
payloads are indistinguishable from a uniformly random byte stream. A examples
of such a transport is obfsproxy~\cite{obfsproxy}, long the go-to obfuscated
transport used by Tor. ScrambleSuit~\cite{scramblesuit} is like obfsproxy in the
content of its payloads, but takes additional steps to obscure its signature of
packet lengths and timing, and is designed to resist active scanning for proxies
(the proxy server says nothing until the client proves knowledge of a shared
secret).

Another path to obfuscation is the steganographic approach: look like
something the censor doesn't block. StegoTorus~\cite{stegotorus}
encodes traffic so as to look like a cover protocol, such as unencrypted HTTP,
using special-purpose encoders.
Code Talker
Tunnel (formerly called SkypeMorph)~\cite{skypemorph} mimics a Skype video call.
FreeWave~\cite{freewave} encodes a digital stream into an acoustic signal
and sends it over VoIP to a proxy which decodes and forwards it.

% fte

Houmansadr et~al.~\cite{parrot} evaluate ``parrot'' systems that imitate a particular implementation of a protocol
and conclude that unobservability by imitation is a ``fundamentally
flawed approach.''
To fully mimic a complex and sometimes proprietary protocol like Skype
is difficult in that the system must mimic not only the protocol operating normally, but also its reaction to errors,
typical traffic patterns, and quirks of its implementation,
otherwise a weak censor can recognize such traffic at a low cost.
Therefore we avoid the parrot approach and use HTTPS as the channel. Geddes et~al.~\cite{acks}
demonstrate that even non-parrot systems may be vulnerable to
attacks that disrupt that covert communication while having no effect or less
effect on legitimate traffic. Their examination is specific to VoIP protocols,
where packet losses and duplication are acceptable. The censor then can
deliberately drop or inject ACKs to disrupt the covert channel without causing
much collateral damage.


There are a few approaches proposed to address the problem of proxy address unblockability.
Tor has long faced the problem of its entry relays being blocked. The list of
relays is public, so it easy to block all of them by IP address. Tor
bridges~\cite{tor-blocking} are relays that are not universally known, intended
to serve as entry points for censored users. A system called BridgeDB seeks to
provide a few bridges to anyone who asks, while at the same time making it
difficult to learn the entire list of secret bridges. The plain Tor protocol
remains relatively easy to detect, so BridgeDB is also capable of distributing
obfsproxy and ScrambleSuit bridges.

Flash proxy~\cite{flashproxy} attempts to address the problem  by
conscripting web users as temporary proxies. Proxies last only as long as a web
user stays on a page, so the pool of proxies is constantly changing and
difficult to block. Their approach to proxy unblockability is in a sense
opposite to this work's: where flash proxy uses many, cheap, unreliable proxies,
we use just one high-value proxy, to block which would cause expensive
collateral damage. There exists a prototype transport that combines flash proxy
with obfsproxy~\cite{obfs-flash}, however it is limited because it is not
possible to obfuscate flash proxy's outermost WebSocket layer.

A technique known as OSS~\cite{oss}, for
``online scanning service,'' resembles ours in certain ways. OSS bounces data
through third-party web services that are capable of making HTTP requests.
Even though such services are capable of fetching a web page, they do not in general
preserve the contents of the page when it is returned to the requestor.
(A translation service, for example,
returns a response after translating it to another language.)
For this reason, OSS does not rely on sending downstream data in HTTP response bodies,
but rather requires the server to make a symmetric reflected HTTP request back to the client.
The main ways in which this work differs is that we are effectively using an OSS that we fully control:
we can count on response bodies being preserved (and so use them to carry
downstream data);
and we make the OSS hard to block by hosting it on an important network resource.

Decoy routing is
a recently proposed approach to making proxies hard to block. Telex~\cite{telex}
relies on friendly ISPs that deploy special software on routers between censors' networks and
popular, uncensored Internet destinations.
Circumvention software is marked with a special ``tag'' that is distinguishable from random only
by Telex routers (not by the censor).
On receiving such a tagged communication, the Telex router shunts it away from its apparent destination.
CensorSpoofer~\cite{censorspoofer}
decouples the upstream and downstream channels. A CensorSpoofer proxy protects its address by
spoofing the source IP address of all packets it sends to the client.
In addition to spoofing, the CensorSpoofer authors propose a SIP-based steganographic channel for downstream data.
Upstream data are carried over a low-bandwidth covert channel such as email.
A recent study~\cite{nodirectionhome} on AS
topology suggests that defeating decoy routing is likely to be expensive for the
censors, if the decoy routers are strategically deployed. Despite that decoy
routing is a sound technical approach, it is still questionable if ISPs are
willing to act against state-level censors. However, the takeaway is that the
censors are unwilling to completely block day-to-day Internet access, which we
can take advantage of.
Of the decoy routing systems, Telex is the most similar to ours.
Both use a piece of network infrastructure as a decoy to redirect certain flows
(for Telex it is an ISP router; for us it is the Google frontend server),
and both tag flows in a way that is visible only to the decoy
(for Telex a tag is a hash embedded in the TLS client randomness; for us it is the HTTP Host header).

% Routing around decoys
% Cirripede

Winter and Lindskog~\cite{foci12-winter} investigated how China's Great Firewall blocks Tor.
They confirmed an earlier discovery of Wilde~\cite{wilde} that the firewall identifies Tor relays using more than passive monitoring:
it actively probes destination addresses to see if they speak the Tor protocol, and if so,
blocks them for the future.
(At the time, the firewall identified connections for future probing by looking for a distinctive
list of ciphersuites sent by a Tor client in its TLS client hello.) %\cite{bug4744}
The discovery of active probing was the motivation for probing resistance in ScrambleSuit.

GoAgent~\cite{goagent} is a direct inspiration for our work in its use of App
Engine and Host header--based domain fronting. GoAgent requires users to upload
a personal copy of the server code to App Engine, and App Engine itself makes
HTTP requests, not some proxy farther out. According to a May 2013 survey~\cite{collateral-freedom},
GoAgent was then the circumvention tool most used in
China, with 35\% of survey respondents having used it in the previous month.
This figure is higher than that of paid~(29\%) and free VPNs~(18\%), and far
above that of other special-purpose tools like Tor~(2.9\%) and Psiphon~(2.5\%).

% CORDON? Say how meek would be classified in it?

\bibliographystyle{plain}  
\bibliography{related-work}

\end{document}
